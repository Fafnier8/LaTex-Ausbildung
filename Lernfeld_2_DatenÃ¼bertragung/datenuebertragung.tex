\documentclass[captions=tableheading,parskip=half, bibliography=totoc]{scrartcl} % KOMA-Script Dokumentenklasse Article [titlepage=firstiscover]

% UTF8 input encoding support
\usepackage[utf8]{inputenc}

% scrhack
\usepackage{scrhack}

% Warnung, falls noch einmal kompiliert werden muss
\usepackage[aux]{rerunfilecheck}

% Paket für Schriftarteinstellung, muss immer geladen werden
\usepackage{fontspec}

% Deutsche Spracheinstellungen, wichtig z. B. für korrekte Trennung
\usepackage[main=ngerman]{babel}
\usepackage[autostyle]{csquotes}

\usepackage{ragged2e}
\usepackage{pdfpages}
\usepackage{pdflscape}
\usepackage{rotating}

% adds Blindtext
\usepackage{blindtext}
\usepackage{longtable}

% Math and Science extensions
\usepackage{amsmath, nccmath}
\usepackage{amssymb}
\usepackage{mathtools}
\DeclarePairedDelimiter\abs{\lvert \,}{\rvert}
\DeclarePairedDelimiter{\bra}{\langle}{\rvert}
\DeclarePairedDelimiter{\ket}{\lvert}{\rangle}
\DeclarePairedDelimiterX{\braket}[2]{\langle}{\rangle}{
    #1 \delimsize| #2
    }

\usepackage[math-style=ISO,bold-style=ISO,sans-style=italic,nabla=upright,partial=upright]{unicode-math}
\usepackage[locale=DE, uncertainty-mode = separate]{siunitx}  %[per-mode=symbol-or-fraction]
\sisetup{math-micro=\text{$\symup{\mu}$},text-micro=$\symup{\mu}$}
\usepackage[version=4, math-greek=default, text-greek=default]{mhchem}
\DeclareSIUnit\Div{DIV}
%
\usepackage{graphicx}
\usepackage{grffile}
\usepackage{float}
\floatplacement{figure}{htbp}
\usepackage[labelfont=bf]{caption}
\usepackage[width=0.75\textwidth]{subcaption}

% Bibliography extensions
\usepackage{booktabs}
\usepackage[backend=biber]{biblatex}
\addbibresource{../lit.bib}

\usepackage{microtype}
\usepackage{xfrac}
\usepackage[shortcuts]{extdash}
\usepackage{expl3}
\usepackage{xparse}
\usepackage{mleftright}
\usepackage[shortcuts]{extdash}

\NewDocumentCommand \delnachdel {mm}
{
    \mathinner{\frac{\partial #1}{\partial #2}}
}
\NewDocumentCommand \delnachdelsq {mm}
{
    \mathinner{\frac{\partial^2 #1}{\partial #2 ^2}}
}
\NewDocumentCommand \dnachd {mm}
{
    \mathinner{\frac{\symup{d} #1}{\symup{d} #2}}
}

\NewDocumentCommand \deln {mmm}
{
    \mathinner{\frac{\partial^#3 #1}{\partial #2 ^#3}}
}

\NewDocumentCommand \leftside {m}
{
\flushleft{#1\;}\justifying
}


\NewDocumentCommand \dif {m}
{
    \,\mathinner{\symup{d} #1}
}

\NewDocumentCommand \zz {}
{
    \mathinner{\mathrm{Z\kern-.3em\raise-0.5ex\hbox{Z}}}
}

\ExplSyntaxOff

% Zum Ausschwärzen von Texten

\usepackage{censor}



% Unterstützung für Links und PDF Metadaten
\usepackage[unicode,pdfusetitle]{hyperref}
\usepackage{bookmark}

%\setmathfont{XITS Math}[range={scr,bfcal}]
%\setmathfont{XITS Math}[range={cal,bfcal}, StylisticSet=1]
\sisetup{math-micro=\text{\mu},text-micro=\mu}

\begin{document}
    \pagenumbering{roman}

    \vspace{2cm}
    
    \title{Datenübertragung}
    
    \vspace{1cm}
    \maketitle
    \thispagestyle{empty}
    \newpage

    \section*{Aufgabe 2}
    \subsection*{2.1}
    Wie lange braucht ein Bit (oder ein elektrisches Signal), wenn es über eine $1000$ Meter lange Kupferleitung mit
    einem NVP von $0.7$ übertragen wird? (rechnen Sie mit einer Vakuum-Lichtgeschwindigkeit $c_0$ von $ \SI{3e8}{\meter\per\second}$)

    \begin{align}
        c &= c_0 \cdot NVP\\
        c &= \frac{s}{t}  \qquad \vert \cdot t \vert:c\\
        t &= \frac{s}{c_0 \cdot NVP}\\
        t &= \frac{\SI{1000}{\meter}}{\SI{3e8}{\meter\per\second}\cdot 0.7}\\
        &\approx 4,77 \cdot 10^{-6} \si{\second}
    \end{align}

    \subsection*{2.2}
    Wie lange braucht ein Bit (oder ein elektrisches Signal), wenn es über eine $1000$ Meter lange Glasfaserleitung mit
    einem Kernbrechungsindex von $n = 1.4$ übertragen wird? 

    \begin{align}
        c &= \frac{c_0}{n} \\
        c &= \frac{s}{t} \\
        t &= \frac{s\cdot n}{c_0}\\
        t &\approx \SI{4.67e-6}{\second}
    \end{align}
    
    \subsection*{2.3}

    Die Übertragung sind relativ gleich schnell .

    \section*{Aufgabe 3}
    Sie wolle eine Datei von ihrem Filserver in Ihrem Betrieb herunterladen.

    \subsection*{3.1}

    Berechnen Sie, wie lange der Dateidownload dauert, wenn die Datei $1 \si{\giga\byte}$ groß ist und über einer $100$
    Base-T-Netzwerkleitung übertragen wird.

    \begin{align}
        t &= \frac{\SI{8}{\giga\bit}}{\SI{0.1}{\giga\bit\per\second}}\\
          &= \SI{80}{\second}
    \end{align}
    Angaben gelten nur bei maximaler Übertragungsgeschwindigkeit.

    \subsection*{3.2}

    An einem Arbeitsplatz in der Marketingabteilung werden durch Videoschnitt große Dateien erzeugt, die innerhalb
    von einer halben Sekunde zum Fileserver übertragen werden sollen. Wie groß muss die Datenübertragungsrate der
    Netzwerkleitung sein? Berechnen Sie die Datenrate bei einer Dateigröße von $\SI{1}{\mega\byte}$, $\SI{10}{\mega\byte}$,
    $\SI{100}{\mega\byte}$ und $\SI{1000}{\mega\byte}$.
    
    \begin{align}
        C &= \frac{D}{t}\\
        \intertext{
            Für $\SI{1}{\mega\byte}$ ergibt das:
        }
        C &= \frac{\SI{1}{\mega\byte}}{\SI{0.5}{\second}}\\
          &= \SI{2}{\mega\byte\per\second}
        \intertext{
            Für $\SI{10}{\mega\byte}$ ergibt das:
        }
        C &= \frac{\SI{10}{\mega\byte}}{\SI{0.5}{\second}}\\
          &= \SI{20}{\mega\byte\per\second}
        \intertext{
            Für $\SI{100}{\mega\byte}$ ergibt das:
        }
        C &= \frac{\SI{100}{\mega\byte}}{\SI{0.5}{\second}}\\
          &= \SI{200}{\mega\byte\per\second}
        \intertext{
            Für $\SI{1}{\giga\byte}$ ergibt das:
        }
        C &= \frac{\SI{1}{\giga\byte}}{\SI{0.5}{\second}}\\
          &= \SI{2}{\giga\byte\per\second}
    \end{align}

    \subsection*{3.3}

    Wie lange braucht ein Datenpaket von $\SI{1000}{\byte}$ Größe, bis es über eine 100-Mbps-Netzwerkleitung komplett gesendet wurde?

    \begin{align}
        t &= \frac{D}{C}\\
          &= \frac{\SI{8000}{\bit}}{\SI{100}{\mega\bit\per\second}}\\
          &= \SI{80}{\micro\second}
    \end{align}

    \subsection*{3.4}

    Sie wollen ein Datenpaket von $\SI{100}{\byte}$ über eine Strecke von Flensburg nach Garmisch-Partenkirchen ($\SI{1000}{\kilo\meter}$)
    übertragen. Wie lange ist das Paket näherungsweise auf einer Leitung unterwegs, bis es beim Empfänger ankommt? Rechnen Sie mit einer Signalgeschwindigkeit
    $c_{\text{Leitung}}$ von $\SI{2e8}{\meter\per\second}.$
    Randbedingung: Wir gehen von der reinen Laufzeit auf der Leitung aus. Router-Durchleitungszeiten u.Ä. werden
    vernachlässigt.

    \begin{align}
        c_{\text{Leitung}} &= \frac{s}{t}\\
        t_{\text{Übertragungszeit}} &= \frac{s}{c_{\text{Leitung}}}\\
          &= \frac{\SI{1000e3}{\meter}}{\SI{2e8}{\meter\per\second}}\\
          &= \SI{5}{\milli\second}
    \end{align}

    \subsection*{3.5}

    Wie lange dauert der gesamte Datentransfer vom Senden des erste Bits bis zum Empfang des letzten Bits? Die Leitung hat eine Übertragunskapazität
    C von $\SI{25}{\mega\bit\per\second}$. Welche Schlussfolgerung können Sie daraus ziehen?

    \begin{align}
      t_{\text{Verbindungsaufbau}} &= \frac{D}{C}\\
        &= \frac{\SI{100}{\byte}}{\SI{25}{\mega\bit\per\second}}\\
        &= \SI{32}{\micro\second}\\
      t_{\text{Übertragungsdauer}} &= t_{\text{Verbindungsaufbau}} + t_{\text{Übertragungszeit}}\\
        &= \SI{5.032}{\milli\second}
    \end{align}

    
    \subsection*{3.6}

    Für die Datensicherung steht einer externe SSD-Festplatte mit USB 3.0-Schnittstelle ($\SI{450}{\mega\bit\per\second}$)
    und eSATA II-Schnittstelle ($\SI{300}{\mega\bit\per\second}$) zur Verfügung. Berechnen Sie die Zeit, die jeweils für
    die Sicherung einer Datenmenge von $\SI{8}{\gibi\byte}$ nötig ist.

    \begin{align}
      t &= \frac{D}{C}\\
      \intertext{
        Für die SSD-Festplatte ergibt sich:
      }
        &= \frac{\SI{8}{\gibi\byte}}{\SI{450}{\mega\bit\per\second}}\\
        &= \frac{8 \cdot 8 \cdot 1024^{3} \si{bit} }{\SI{450e6}{\bit\per\second}}\\
        &\approx \SI{152.71}{\second}\\
    \end{align}

    \subsection*{4 Berechnen Sie die Fahrstrecke eines Autos zwischen dem Absenden einer Nachricht und dem Empfang der Antwort.}
    
    


\end{document}
